\renewcommand{\baselinestretch}{1.5}
\fontsize{12pt}{13pt}\selectfont

\chapter[摘要]{摘~~~~要}
\markboth{摘~要}{摘~要}

刀具磨损量的精准监测和预报对于保证产品加工质量、提升生产力和降低成本具有非常重要的意义。本文对于加工时产生的信号进行分析,建立基于刀具磨损状态的监测和退化模型,为刀具的高效应用和产品质量的保证提供依据。本文完成的主要研究工作如下:

\begin{enumerate}
	\item 单工况下基于\gls{resnet_zh}的刀具磨损在线监测和基于\gls{rnn_zh}的未来预报\\
	采集加工时由三轴力、三轴加速度和一个声发射传感器所采集的信号,进行小波变换并作为\gls{resnet_zh}的输入数据,构建在单工况下的刀具在线磨损量监测模型,从而监测实时磨损量。分别将历史和未来磨损量进行编码,构建基于刀具磨损量退化的基于\gls{lstm_zh}单元的\gls{rnn_zh}模型,从而实现较宽范围内的刀具未来磨损量预报。
	\item 多工况下刀具磨损数据修复,在线监测及其迁移学习技术\\
	使用过拟合的\gls{rf_zh}针对多工况下的数据进行标签修复。采集不同工况下的加工工件信号和对应的工况信息,使用相应的编码方式作为\gls{resnet_zh}的隐空间,从而实现由一个模型针对多种工况的综合在线监测技术。
	\item 基于\gls{tda_zh}的深度学习模型的可视化技术\\
	使用\gls{tsne_zh}针对\gls{resnet_zh}中高维卷积核进行\gls{pca_zh},接着使用\gls{dbscan_zh}和\gls{hdbscan_zh}针对降维后的主成分进行\gls{tda_zh},从而获得核心特征的分布规律,真实的反应卷积层工作过程,并且能够评估深度学习模型的性能,提高了现有深度学习模型的学习透明性、可视化和可解释性。
\end{enumerate}

通过实验数据验证和与同行的对比,本论文提出的方法可以非常准确的监测刀具实时磨损量,并且能够复用多个工况下的信号特征进行监测,同时首次完成了针对刀具磨损未来量的多尺度时间下的预报问题,对深度学习模型进行了分析,揭示了其工作过程和特点。
\vspace{-10pt}

\vspace{1em}
\noindent {\fHei 关键词:} \quad 刀具磨损量, 深度学习, 在线监测, 未来趋势预报, 可视化技术

\clearpage
\endinput